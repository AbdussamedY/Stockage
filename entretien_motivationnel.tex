\subsection{Entretien motivationnel (coefficient 2)}

\subsubsection{Déroulement de l'entretien}

L’entretien motivationnel, comme l’indique si bien son nom, a pour but d’apprendre à te connaître, de voir ta motivation et tes envies. Il est l’entretien le plus important des quatre, comme le montre son coefficient.\\

Toutefois, cela ne veut pas dire qu’il faut le voir comme une épreuve et changer sa personnalité tout entière pour cet oral. Il s’agit d’être authentique et de faire sortir ta curiosité naturelle qui t'a amenée jusqu’ici. Si tu as vraiment envie d’intégrer le cursus et que tu as bien compris les enjeux de ta candidature, le jury le ressentira si tu restes toi-même.\\

L’épreuve comporte deux parties : une première d’\textbf{analyse d’article} et une deuxième d’\textbf{entretien sur tes motivations}. Il s’agit du seul oral où le jury est composé de plusieurs personnes. Tu seras ainsi face à 5 chercheurs et médecin-chercheurs, dont Alain Bessis, le directeur du programme Médecine-Sciences.\\

Dans un premier temps, un résumé d’un article scientifique (souvent son abstract) t'est fourni. Tu as 15 minutes dans une salle à part et sous surveillance pour réfléchir à ce résumé et préparer un petit point sur celui-ci. Puis on t'amène devant le jury et l’épreuve commence. On te demande tout d’abord de résumer rapidement l’article en question et une discussion sur l’intérêt de cette découverte, les méthodes utilisées et les limites éventuelles commence entre toi et le jury. Cette première partie correspond à 10 minutes de l’épreuve. Pour s’entraîner à cet exercice, le mieux est de lire beaucoup d’articles scientifiques, voire de s’entraîner à les analyser, puis de s’entraîner seulement à partir de l’abstract à imaginer quelles expériences peuvent être proposées et ensuite de vérifier ce que les auteurs ont réellement fait en lisant la suite.\\

Les 20 minutes suivantes sont dédiées à des questions sur ta motivation. Généralement, on te demande de te présenter et d’expliquer pourquoi tu es là. Puis des questions assez classiques sur ton CV et tes projets professionnels suivent. Tu retrouveras des questions plus ou moins redondantes dans les témoignages juste en-dessous, mais le jury s’adapte aussi évidemment aux particularités de ton dossier et à ce que tu lui expliques pendant l’entretien.

\subsubsection{Témoignages}

\lettrine{{\color{violet} \oldpilcrowfive}}{}
C’était un article sur la synaptobrévine, molécule transmembranaire de vésicules synaptiques. J’ai
présenté de manière classique puis ils m’ont posé des questions auxquelles j’ai plus ou moins bien
répondu. Il s’agit, je pense, de réfléchir à voix haute et quand on ne sait pas de repartir de nos
connaissances sur le sujet (ici, le fonctionnement normale de la synapse…).
Ensuite, ils ont commencé par me demander de me présenter, dire pourquoi je suis là. Puis, ils m’ont
demandé si j’étais en stage et comme c’était le cas je leur ai expliqué. Ils rebondissaient beaucoup
sur ce que je disais, me demandaient de préciser. Ils m’ont aussi demandé : les qualités d’un
chercheur et d’un médecin, comment je voyais le lien entre la médecine et la recherche, pourquoi
j’irais à l’ENS à Paris alors que je pourrais aller à l’ENS de Lyon et aussi l’école de l’Inserm.
Le plus important, je pense que c’est d’être honnête et de montrer vos vraies motivations parce que
c’est là que vous serez convaincant (et pas la peine d’être super conventionnel, vous pouvez être
motivés par autre chose que par la perspective d’être PU-PH). Je pense qu’ils sont intéressés par des
étudiants qui sont motivés/passionnés dans leur vie (quel que soit l’objet de cette passion -ou
presque) et enthousiastes. En plus, être honnête vous donne de la légitimité pour la suite : si vous ne
dites pas la vérité et qu’en arrivant en M1 l’année prochaine vous dites « ah mais non c’est pas du
tout ça j’avais juste menti pour l’oral », c’est pas terrible terrible et ça risque d’être un problème
aussi pour vous car ce qui vous attendra ne vous plaira pas !\\

\lettrine{{\color{yellow!80!black} \oldpilcrowfive}}{}
On m'appelle, je me mets dans une salle tout seul, accompagné d'une personne pour me surveiller $\tiny\text{(je suis sage pourtant...)}$. Les 15 minutes sont lancées : je suis face à un abstract d'article à propos de la synaptobrévine (hop, ta réaction \href{https://www.youtube.com/watch?v=6elK8VI1rPs}{<ici>}). En bref, les chercheurs mettent en évidence que les neurones ne communiquent pas entre eux uniquement via les synapses, et qu'il existe donc une autre voie de communication intercellulaire. Elle fait intervenir des vésicules de sécrétion, sécrétées par des interneurones et captées par les cellules cibles justement grâce à la synaptobrévine.\\
Les propos tenus par les chercheurs implicitaient plusieurs méthodes, qu'il était assez facile de deviner. Je suis parvenu à comprendre l'ensemble de l'abstract en 15 min, si ce n'est \textbf{qu'une seule partie} d'une phrase. J'ai pourtant forcé la compréhension, je l'ai lue et relue... Mais ça ne voulait pas...\\
Puis le glas a sonné... J'entre dans une salle, le jury se tient assis devant moi. Ils sont 5, ils me regardent. Ils avaient l'air siiiii sympas !!! $\tiny\text{(ouais bon... je stressais donc j'ai gardé ça pour moi)}$.\\
Je présente l'article. Tout se passe au mieux. Puis je m'approche de la fin, et je le sais, ils le savent, vous le savez : j'ai sauté une partie volontairement (celle que je n'ai pas très bien comprise). J'ai préféré rester honnête en leur disant que je n'arrivais pas à capter le message qu'ils cherchaient à me faire passer. A ce moment, Alain Bessis et un autre membre du jury essayent de m'aider. Ils voient que je donne de bons éléments mais que le déclic n'y est pas encore... Ils décident de passer à la suite, en me demandant de me présenter (mais j'ai continué à cogiter dans ma tête).\\
Spoiler alerte : $\textcolor{red}{\text{reste honnête}}$, ils sont beaucoup trop forts, ils arrivent à savoir ce qu'on a dans la tête. Tu risqueras de ne pas être cohérent.e dans tes propos, ce qui pourra te nuire.\\
Après que je me sois présenté brièvement, ils me posent des questions parfois assez ouvertes, parfois très précises. Ils demandent mon avis sur le rôle que peut avoir un médecin-chercheur ; ils me demandent en quoi l'ENS pourrait m'aider à atteindre les objectifs que j'ai listés dans ma lettre de motivation. Enfin bref, reste toi-même. Il faut simplement leur montrer qui tu es, avec toutes les qualités que tu peux avoir.\\
Alors que le jury est sur le point de clôturer l'entretien, j'ai enfin eu le déclic ! Je leur ai donné mon raisonnement, et j'ai senti que j'avais marqué un bon point. Ca montre bien ce qu'on essaye de te dire. On n'est pas là pour lister toutes nos connaissances. Ils le savent autant que toi, tes connaissances ont une limite. Ce qui les intéresse réellement, c'est ta capacité à te débrouiller avec ce que tu as.\\
Prend du plaisir dans cet entretien, le jury est là pour se concentrer sur ta personne. Il n'est pas là pour te piéger, ou te mettre en difficulté ; il ne cherche qu'à te comprendre !\\
\newpage
\lettrine{{\color{violet} \oldpilcrowfive}}{}
2e oral, juste avant le déjeuner. Grosse pression sur l'entretien de motivation (on t'explique que c'est le seul qui compte vraiment, que les autres c'est juste pour la forme, mais j'avais vraiment pas l'impression de m'être entraîné, d'avoir bossé des questions types...).\\

Article préparé : abstract d'un Nature 2021, \href{https://doi.org/10.1038/s41586-021-04267-8}{\textit{<Human blastoids model blastocyst development and implantation>}}$\footnote{https://www.nature.com/articles/s41586-021-04267-8}$.\\

Je m'étais très peu préparé à l'analyse d'abstract, j'avais lu quelques articles mais ne m'étais pas du tout appliqué sur la forme de l'entretien : en conséquence, je parle trop vite, trop fort, et beaaaaucoup trop longtemps. L'article était super intéressant et j'avais trouvé plein de méthodes potentielles et de limites de l'article, à tel point que le jury a dû m'arrêter 2 fois avant ma conclusion, j'ai dû bacler cette dernière et ai un peu regretté de ne jamais m'être chronométré. \\

Dans ma panique, j'ai répété 5 fois que ERK et TGF-$\beta$ étaient des facteurs de transcription, et avais complètement omis le mot syncytiotrophoblaste (un peu gênant en embryo) car je n'arrivais pas à le retrouver. Ca n'a pas manqué, c'étaient les 2 premières questions du jury. J'ai plus ou moins réussi à rattraper mes erreurs, et leur ai dit honnêtement que j'avais oublié le mot "syncytiotrophoblaste", mais que je pouvais leur raconter à quoi ça servait. \\

L'un des membres du jury, qui est celui qui avait lu l'article, a relevé quelques autres imprécisions que j'avais faites (j'avais dit que le trophectoderme était intraembryonnaire, explication du lien entre la voie HIPPO et tumeurs \textit{(jonction cadhérine-cadhérine ?)}).\\

Comme j'avais déjà pris mon temps, la partie analyse scientifique fut assez courte, et on est rapidement passé aux questions de motivation. J'étais globalement assez déçu, j'avais lu dans les OaN qu'il y avait une véritable discussion scientifique avec le jury, passionnée et enjouée. J'ai plus eu l'impression d'avoir des questions bateau, qui m'ont empêché de vraiment dire ce que je voulais. J'avais l'impression que la discussion était assez fermée, le jury n'était pas très réactif \textit{(peut-être avaient-ils hâte de manger ?)} et passait juste ma lettre de motivation à la moulinette, chaque ligne étant sujette à une question de leur part pour bien me faire comprendre que j'avais été très maladroit. \\

Pour répondre à toutes ces questions pendant une vingtaine de minutes, j'ai pris la décision de rentrer dans leur jeu, et de me foutre de ma gueule avec eux. Je ne sais pas si c'était le choix le plus heureux, mais j'ai eu l'impression d'avoir légèrement détendu l'atmosphère, pour peut-être laisser place à une vraie discussion ? Je ne l'ai malheureusement pas eue, mais cela m'a permis de profiter un peu plus de cet entretien.\\

\uline{Questions types :} qu'est ce que vous venez faire à l'Ecole ? Pourquoi êtes-vous à cet entretien ? Que faites-vous en stage en ce moment ? Avez-vous des envies pour plus tard, un domaine qui vous intéresse tout particulièrement ? Qu'est ce qu'un bon chercheur selon vous ? Thèse précoce ?\\

En sortant de cet oral, j'avais juste l'impression d'avoir répondu à leurs questions, sans accrocs mais sans rien de transcendant non plus, rien qui ne me démarque. Je suis donc rentré et ai essayé de me concentrer sur les oraux du lendemain. \\

\lettrine{{\color{yellow!80!black} \oldpilcrowfive}}{}
Pour moi c’est l’épreuve qui s’est le moins bien passé.\\
Bon pour l’article \textit{(je ne me rappelle même plus de quoi ça parlait, c’était de la physique)}, je n’ai pas été
brillante mais je me suis dépatouillée. Je ne suis pas sûr qu’on puisse vraiment réviser pour ça…\\
Puis pendant l’entretien, ils ont cité à plusieurs reprises des choses que j’avais écrites dans ma lettre de motivation en me demandant de me justifier. J’avais quand même écrit dans ma lettre de motivation que je voulais être astronaute et faire de l’astrobiologie... \\
Puis je leur ai parler du stage que j’étais en train de faire et ce qui me plaisait. Mon stage était surtout orienté bioingénieurie. Et à partir de ce moment j’ai eu le droit à 3 fois la même question (c’est beaucoup en l’espace de 5 min (et re hop voilà ta réaction \href{https://www.youtube.com/watch?v=6elK8VI1rPs}{<ici>} $\tiny \text{(et re re hop voilà ta ré... ok ok continuons)}$) : « Nous on ne fait pas de la bioingénieurie, on fait de la recherche fondamentale, ce n’est pas la même chose non ? Quel est le lien entre la bioingénieurie et la recherche fondamentale etc. ». \\
Je suis restée assez zen, je ne me suis pas faite déstabilisée et j’ai défendu le fait qu’il faut des deux pour faire avancer la science et, qu’en quelque sorte, l’un ne
va pas sans l’autre.\\
J’étais assez déçue parce que je trouvais qu’ils n’avaient pas assez exploré ma personnalité… mais bon je pense que ce qui est clef dans ce cas, c’est d’être confiant, de leur sourire et de leur dire « j’entends votre remarque mais… ».\\

Sinon les examinateurs ont l’air bienveillants, donc je n’ai pas été trop impressionnée d’en avoir 4
devant moi (surtout après l’entretien de l’ENS de Lyon où ils étaient au moins 10).\\

Courage à vous, ce n’est pas une période facile mais ayez confiance en vous !!!!!!\\

\lettrine{{\color{violet} \oldpilcrowfive}}{}
Mon article portait sur l’indépendance de la production de neurotransmetteurs vis-à-vis du cycle circadien (ou un truc dans le genre, mais c’était de la neuro).\\
L’interaction lors de l’analyse d’article se fait de manière quasi-exclusive avec un seul interlocuteur, qui est le seul à avoir lu l’article. Rien de bien intéressant à noter sur cette partie, si ce n’est que ce qui est valorisé est la démarche de réflexion et la capacité à rebondir sur les questions posées (comme sur les autres épreuves en gros). \\
Sur la partie motivation, j’ai eu l’impression que le jury joue au « good cop bad cop », tout en restant très bienveillant. Les questions sont assez variées, allant des classiques « qu’est-ce qui fait selon vous un bon chercheur ? », « et un bon médecin ? » jusqu’à des questions assez bizarres du type :

\begin{center}
\begin{minipage}{0.8\linewidth}
« Vous dîtes dans votre lettre que vous êtes créatif, avez-vous des exemples ? \\
- EUUUUUU je joue de la guitare et du violon et j’aime le bricolage XD, \textit{dit-il d'une voix madrilenement marseillaise}\\
– Ah, et en quoi jouer de la guitare est créatif ? »\\

(petit moment PLS, mais à nouveau le plus important est de rebondir donc j’ai répondu que je faisais de l’impro, \href{https://www.youtube.com/watch?v=v1VDtTgh05k}{cela étant totalement faux)}.
\end{minipage}
\end{center}

Cela étant dit, \href{https://drive.google.com/file/d/1F_3OgUe7bgE-k7FpomonOPUH-EZ-9PY5/view?usp=share_link}{il est cependant important de rester honnête et soi-même}, car le jury semble avoir une capacité surnaturelle à scruter votre âme et ira chercher les incohérences dans les moindres recoins de votre dossier et discussion (d’où l’importance de faire une lettre de motivation honnête et ne rien dire que ne pourrez pas défendre à l’oral). À la question : « pourquoi la recherche ? » e répondez pas « j’aime la bio » aussi vrai soit-ce, j’ai commencé ma phrase avec ça (en guise d’introduction) et ils ont tous soufflé en haussant les sourcils. \\
Conseil de la fin, même si vous avez l’impression de ne pas avoir réussi ce n’est pas forcément le cas. Perso, j’étais allé au concours sans trop savoir ce que je faisais là et je me suis étouffé pendant l’entretien (un membre du jury a dû aller me chercher un verre d’eau) et finalement j’ai eu 17. Donc c’est normal d’avoir des doutes sur votre parcours, vous n’êtes qu’en P2, et de discuter avec le jury des différentes options de parcours qui vous attirent témoigne de votre maturité et sincérité ! Courage pour vos candidatures :)\\

\lettrine{{\color{yellow!80!black} \oldpilcrowfive}}{}
Pour la partie analyse d’abstract, l’article portait sur la découverte d’une bactérie Thiomargarita qui fait 1 cm de long, ayant des vésicules cytoplasmiques contenant de l’ADN et des ribosomes.\\
Le jury a beaucoup apprécié la mise en perspective de cette découverte de vésicules étant finalement des mini-unités de traduction qui pourraient être utilisés pour en faire un outil d’expérimentation en biologie, voire un vecteur thérapeutique étant plus facilement transportables et diffusibles que de l’ADN, et pouvant suppléer des gènes déficitaires sans aller interagir avec le génome nucléaire.\\
Enfin, le jury a posé quelques questions dont notamment une à laquelle je n’ai vraiment pas su quoi répondre et comment argumenter : pourquoi une bactérie de 1 cm c’est extraordinaire, qu’est ce qui, normalement, limite la taille maximale d’une bactérie ? Le réponse était le rapport surface/volume et donc les implications qui en dérivent en termes de besoins VS possibilités d’interaction avec l’extérieure.\\
Enfin pour la partie entretien motivationnel, on m’a laissé expliquer pourquoi je voulais faire ce cursus, ce à quoi j’ai pu répondre en exposant mes motivations générales et pour l’ENS plus particulièrement. Ensuite, m’ont été posées des questions plus précises sur pourquoi j’aimais les sciences, quel intérêt d’être médecin-chercheur et des questions un peu déstabilisatrice du style "pourquoi n’avez-vous pas parlé dans votre lettre de motivation de cette chose figurant dans votre CV ?".\\

\lettrine{{\color{violet} \oldpilcrowfive}}{}
J’ai eu un article sur un gène qui a été identifié chez la girafe, qui permet de réguler sa pression artérielle pour l’adapter à sa « morphologie exceptionnelle » : \href{https://pubmed.ncbi.nlm.nih.gov/33731352/}{<ici>}\footnote{https://pubmed.ncbi.nlm.nih.gov/33731352/}.\\
Si je me rappelle bien, ils ont commencé par une question très générale : « Qu’avez-vous pensé de l’article ? »\\
Ensuite ils m’ont demandé "comment ont-ils fait pour éditer le génome de la souris ?" ; j’ai d’abord pensé à une transfection virale du gène FGFRL1 de la girafe. Ils m’ont laissé développer puis m’ont dit que ce n’était pas judicieux, car la souris avait déjà son propre gène FGFRL1. J’ai répondu « Ah, bah on fait un crispr alors ! ». C’était la bonne réponse.\\
Ils ont ensuite voulu me poser une question sur une technique, sur la préparation de cellules, je ne sais pas trop, je lui ai demandé de repriser car je ne voyais pas où il voulait en venir, mais Alain Bessis lui a dit « Non mais quand même c’est très dur ce que tu demandes » et du coup il a abandonné l’affaire. Donc je ne saurais jamais de quoi il parlait...\\
Enfin, ils m’ont demandé en quoi cette découverte pouvait-elle être importante pour identifier des traitements contre l’hypertension chez l’Homme.\\

Pour la partie « motivation », c’est bien d’avoir relu son CV et sa lettre de motivation car ils vous posent des questions dessus :

\begin{itemize}
    \item D’où / De quelle fac je venais ?
    \item Est-ce que j’étais en stage en ce moment ?
    \item Comme c’était le cas, qu’est-ce qu’y s'y faisait etc... Comme c’était des sciences cognitives, et que c’était pas trop leur domaine, ils ont moyennement compris, et j’ai même dû ré-expliquer à un moment. Mais c’est important de montrer que vous savez reformuler les choses et leur montrer que vous avez bien compris ce que vous faites. Donc ça a pris un petit moment.
    \item Qu’est ce que je voulais faire comme recherche ?
    \item Dans ma lettre de motivation et mon CV, j’y ai dit que j’avais été admissible au concours de médecin militaire, ils m’ont demandé pourquoi j’avais voulu faire ça.
    \item Je crois qu’ils m’ont demandé comment j’avais découvert l’ENS.
    \item Dans ma lettre j’ai écrit que « je ne me posais pas de barrière, dans ma scolarité », une examinatrice m’a demandé ce que je voulais entendre par là. Pour justifier mes dires, j’ai pris l’exemple de mon stage, qui se déroulait à Marseille, alors que j’aurais pu choisir la facilité en restant sur Bordeaux, mais ce que je voulais faire était à Marseille etc...
    \item Ils m’ont demandé pourquoi je voulais faire l’ENS, qu’est ce que ça m’apporterait de plus par rapport à mon double-cursus actuel ? J’ai parlé de la qualité et de la quantité d’enseignements, surtout du choix à la carte qui offre une liberté. L’émulsion intellectuelle et le partage des cultures. J’ai surtout insisté sur l’interdisciplinarité, que je voulais « retrouver ».
    \item Donc ils m’ont demandé pourquoi est ce que je disais « retrouver » ? J’ai expliqué que quand on est au lycée, on fait encore plein de matières et donc une certaines richesse, et que progressivement, en avançant dans nos études, notamment de médecine, on ne fait plus que de la médecine, et que les autres disciplines me manquaient, la richesse intellectuelle etc... Et que l’ENS, grâce aux cours hors département de bio, allaient me permettre de retrouver cette interdisciplinarité.
    \item J’avais aussi dit que j’étais intéressée par la « Space Health Week » qui se déroule à Cologne, comme j’étais intéressée par la médecin spatiale, mais M. Bessis m’a reprise en insistant que ce n’était pas sûre qu’elle aurait lieu, et que je ne pouvais seulement reposer ma volonté de venir à l’ENS pour une semaine. Ils essaient de vous bousculer, il ne faut surtout pas se démonter. J’ai répondu que ce n’était pas le cas et que je considérais cette semaine comme du « bonus ».
    \item On m’a également demandé si je n’avais pas fait médecine, qu’est ce que j’aurai fait comme études. J’ai expliqué que je ne m’étais jamais vraiment intéressée aux prépas, mais que lors de mes révisions pour le concours, je m’étais rendu compte qu'une prépa BCPST aurait bien pu me plaire et me correspondre. C’est une réponse qui a eu l’air de leur plaire. Après tout, les 2/3 des bio arrivent de BCPST.
    \item Enfin, ils m’ont redemandé quel était mon classement en P1 car il n’avait pas compris sur la feuille (vive la réforme qui rend tout compliqué !)
\end{itemize}

Faites donc bien attention aux mots / expressions que vous employez dans votre lettre / entretient. Ils essayent de vous déstabiliser, c’est normal. Relisez bien votre lettre. Soyez le plus clair possible quand vous parlez. Montrez que vous êtes motivés du début à la fin !\\
Je ne me rappelle pas avoir eu de questions dessus, mais c’est bien d’avoir une idée des cours que vous voudriez prendre en bio et ailleurs à l’ENS, ils peuvent vous le demander.\\
Apparemment, ils ont noté à la fin de mon entretien que « j’avais la tête dans les étoiles mais que je gardais les pieds sur Terre », ce qui leur a plu.\\


\lettrine{{\color{yellow!80!black} \oldpilcrowfive}}{}
\textbf{Partie article :} \href{https://doi.org/10.1038/s41586-022-04641-0}{<Skin cells undergo asynthetic fission to expand body surfaces in zebrafish>}\footnote{https://doi.org/10.1038/s41586-022-04641-0}\\

Lors de la préparation, j’ai commencé par une première lecture de l’abstract, qui m’a semblé très clairement découpé (ce qui n’est vraiment pas le cas de certains abstract que j’avais lu) en hypothèses, méthodes et résultats. Pour gagner du temps j’ai donc surligné l’abstract en suivant cette structure et indexé chaque grande partie dans mon brouillon en explicitant les méthodes de façon précise et en essayant de me représenter en détail les expériences qu’ils ont pu mener. J’ai essayé de trouver quelques limites aussi. 

\begin{itemize}
    \item Comment pensez-vous qu’ils ont pu créer un système de marquage multicolore ? 
\end{itemize}

J’avais déjà vu ce genre de marquage (thanks to brainbow), et j’avais dessiné pendant ma préparation en quatre coups de surligneurs le système Cre-Flex et Cre-Lox qui permet cela (ne perdez pas votre temps de préparation à dessiner des schémas). J’étais assez confiante face à la réaction des examinateurs, qui ont posé pas mal de questions comme suit. 

\begin{itemize}
    \item Comment on fait s’exprimer ce type de construction génétique dans la cellule ? 
\end{itemize}

J’ai parlé de l’emballage dans des lentivirus, de CRISPR-Cas9, de la présence de promoteurs spécifiques qui peuvent par exemple être sensibles aux oestrogènes et permettre une induction à un moment souhaité…

\begin{itemize}
    \item Mais l’oestrogène ne va-t-elle pas perturber le développement ? 
\end{itemize}

Exemple de question piège, à laquelle j’ai répondu en disant que c’est possible mais qu’il doit y avoir une dose seuil n’induisant pas d’effets sur le développement et que cela a dû faire l’objet de recherches précédentes. Un des examinateurs a affirmé cela et a pris le relai sur les questions.  

\begin{itemize}
    \item Quel est l’intérêt d’étudier l’expansion surfacique durant le développement ? 
    \item En quoi la viabilité des cellules à génome réduit est-elle dérangeante ? 
\end{itemize}

J’avais évoqué le fait que c’était étonnant qu’ils trouvaient que jusqu’à 50\% des cellules qui se divisaient avaient un génome de taille réduit (car elles ne se répliquaient pas avant) en termes de viabilité de ces types cellulaires, d’où la question. Je me suis dit que si ces cellules étaient non viables et mourraient avant que la surface n’ait pu s’expandre, cela ralentissait la croissance (mais il s’agissait en fait de cellules à différenciation terminale). J’ai aussi expliqué que cela pouvait être à l’origine de pathologies par dérégulation de l’expression génique, et j’en ai profité pour dire que ce serait une bonne méthode pour savoir quelles gènes étaient « essentiels » à la survie de la cellule en séquençant lesdites cellules à génome réduit. J’ai aussi évoqué la desquamation que l’on connaît physiologiquement chez l’homme, qui pourrait s’appliquer à ces cellules comme processus de régulation du développement normal. 

\begin{itemize}
    \item Est-ce que l’on peut regarder les forces de tension et de traction pendant le développement ? 
\end{itemize}

J’ai évoqué les canaux PIEZO, et Adrem Patapoutian à l’occasion, en disant qu’ils étaient mécanosensibles et qu'ils permettraient de rendre compte de ces forces de traction et de tension pendant le développement. J’ai donc proposé d’effectuer un KO de ces gènes à certains moments du développement par système Cre et oestrogènes ou Cre inductible par un promoteur qui s’exprime à un moment donné du développement et que l’on veut étudier. Cela aurait permis d’avoir une idée assez précise de leur rôle dans le développement, et par les déformations observées des mutants avoir une idée des forces exercées normalement. \\

Globalement les questions étaient très intéressantes, et semblaient émaner spontanément du jury comme s’ils voulaient en discuter avec moi pour savoir ce que j’en pensais plutôt que pour strictement tester mes connaissances, ce qui a mené vers un échange enrichissant. Revoyez bien les méthodes de base en biologie moléculaire, et sachez les expliciter et savoir pourquoi on les utilise. 

\textbf{Partie motivation :}  J’étais vraiment très motivée par l’idée d’entrer à l’ENS, j’ai donc assez naturellement préparé cette épreuve en réfléchissant aux fondements de ma motivation ce qui m’amenait par exemple à me perdre sur des sites de vulgarisation scientifique ou de présentation de chercheurs, de conférences, de laboratoires… Néanmoins j’ai voulu prendre le temps de répondre précisément aux questions qui revenaient souvent lors de cet oral, ne voulant pas m’égarer et risquer d’être piégée (par exemple en citant tel chercheur sans avoir un exemple concret de travaux qu’il a mené, ou en disant que tel concept me plaît sans dire pourquoi ni savoir citer de laboratoires qui travaillent sur le sujet). Finalement en faisant ce travail d’approfondissement, j’avais les idées claires et j’étais encore plus motivée, donc je ne peux que vous le conseiller. \\
Dans la même optique j’avais pris le temps de travailler ma lettre de motivation, chose que les examinateurs m’ont fait remarquer à la toute fin de mon oral en me posant une question assez déstabilisante : « Votre lettre suit tout un cheminement pour nous convaincre de vous prendre, mais pourquoi avoir fait le choix de consacrer tout un paragraphe à qui vous êtes ? ». C’est comme s’ils avaient cerné toute mon hésitation à écrire ce paragraphe, où je me présentais, parlais de mon enfance, de mes activités extra-scolaires et notamment associatives, ou de ma participation au tutorat des PASS. J’ai répondu en disant en gros que je pensais qu’ils devaient tout autant regarder « qui nous sommes » que « ce dont on est capable ». Autrement l’oral a débuté par la question classique : 
\begin{itemize}
    \item Pourquoi êtes-vous là ?
\end{itemize}

Je crois que je me suis emportée sur une tirade, ce qui m’a valu être coupée pour me rediriger vers des questions plus précises. Peut-être avaient-ils senti que j’avais préparé la réponse à ces questions qui reviennent tous les ans. J’ai ensuite dû justifier certaines formulations dans ma lettre, ajouter certains exemples à des passages où il en manquait. 

\begin{itemize}
\item Pouvez-vous donner un exemple d’avancée permis par les organoïdes ou l’optogénétique (que je citais dans ma lettre) ? 
\end{itemize}

Evidemment vous pouvez vous préparer à ce que l’on vous pose ces questions et faire le choix d’en parler dans votre lettre. Au moment de la rédaction je n’étais pas allée si loin pour toutes les phrases que j’écrivais (et je pense qu’ils cherchent aussi à voir comment vous réagissez et si vous avez des exemples sous la main). Vous comprenez en revanche que votre lettre conditionne pas mal votre oral de motivation : j’avais essayé de répondre dans ma lettre aux questions qui revenaient dans les entretiens des années précédentes, ce qui m’a valu de ne pas les avoir à l’oral et d’avoir plutôt des questions inattendues. J’ai donc pu avoir une conversation avec les examinateurs sur d’autres points très intéressants, comme ma vision de la médecine, du rôle de médecin et de son articulation avec la recherche. Je vous conseille en tous cas de bien relire votre lettre avant l’oral, et de réfléchir aux parties qui pourraient susciter des questions des examinateurs. 

\newpage