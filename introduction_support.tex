\subsection{Introduction et supports de travail}

Une fois l’étape des dossiers passée, les oraux pour l’admission au cursus Médecine-Sciences sont au nombre de quatre : biologie (coefficient 1,5), chimie (coefficient 1), physique (coefficient 1) et un entretien motivationnel (coefficient 2).\\

Voici un tableau récapitulatif des programmes et des supports que nous conseillons pour la préparation :\\

\begin{tabularx}{15cm}{|X||m{6cm}|m{6cm}|}
\hline
&\textsc{Programme officiel}&\textsc{Supports conseillés}\\\hline\hline

\textsc{Biologie}&Programme de la première année de médecine et certains paragraphes du programme de BCPST (à l’exception de la biologie végétale)&Livre de BCPST (Biologie tout-en-un, Peycru, Dunod) en com- plétant avec des exemples des cours de première année de médecine qui sont plus détaillés\\\hline

\textsc{Physique}&Programme de l’enseignement de spécialité de physique-chimie en terminale à l’exception des capacités expérimentales&Cours de la première année de médecine (Approche physique du transport de la matière dans les milieux biologiques, Urbach, Ellipses, Exercices corrigés de Feynman, la physique dans le mille pour la cinétique et les référentiels) avec des exercices de BCPST ou de PCSI et éventuellement un livre de terminale.\\\hline

\textsc{Chimie}&Programme adapté de l’ancien programme de PASS avec des éléments du programme de terminale et de spécialité physique-chimie du lycée&Cours de première année de médecine (Chimie fondamentale, Chottard, Editions Méthodes) en s’entraînant avec des livres de BCPST ou PCSI (Le Lavoisier PCSI avec des exos types oraux des grandes écoles), les cours de Paul Arnaud. Eventuellement un livre du programme de terminale. Sites d’exos : \href{http://chimie-pcsi-jds.net/}{http://chimie-pcsi-jds.net/}\\\hline
\end{tabularx}

\newpage